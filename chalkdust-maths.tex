\documentclass{article}
\usepackage{maths}
\usepackage{comment}

\newenvironment{doctable}{
    \begin{tabular}{|l|l|l|}
    \hline
    Command&Description&Example\\
    \hline
    \hline}{\end{tabular}}
\newcommand{\docline}[3]{\texttt{\detokenize{#2}}&#1&\texttt{\detokenize{#3}} gives $\displaystyle#3$\\\hline}

\title{Chalkdust v\input{VERSION}}
\author{Matthew W.~Scroggs \& Adam K.~Townsend}
\begin{document}
\maketitle

\verb@chalkdust-maths@ is a package for \LaTeX{} that provides common mathematical commands. To use it, put \verb@\usepackage{chalkdust-maths}@
at the top of your TeX file.

\section{Vector calculus operators}

\begin{doctable}
    \docline{Del operator}{\del}{\del}
    \docline{Grad operator}{\grad}{\grad}
    \docline{Vector del}{\vdel}{\vdel}
    \docline{Vector grad}{\vgrad}{\vgrad}
    \docline{Cross product}{\cross}{\cross}
    \docline{Laplacian}{\Lap}{\Lap}
\end{doctable}

\section{Vectors, tensors, etc}

\begin{doctable}
    \docline{Vector}{\v}{\v{x}}
    \docline{Matrix}{\m}{\m{A}}
    \docline{Tensor}{\t}{\t{A}}
    \docline{Basis}{\b}{\b{B}}
    \docline{Upright tensor}{\tu}{\tu{A}}
\end{doctable}

\section{Operators}
\begin{doctable}
    \docline{Real part}{\Real}{\Real}
    \docline{Imaginary part}{\Imag}{\Imag}
    \docline{Cosecant}{\cosec}{\cosec}
    \docline{Hyperbolic cosecant}{\cosech}{\cosech}
    \docline{Hyperbolic secant}{\sech}{\sech}
    \docline{Sign}{\sgn}{\sgn}
    \docline{Error function}{\erf}{\erf}
    \docline{Complementary error function}{\erfc}{\erfc}
\end{doctable}

\section{Matrix operators}
\begin{doctable}
    \docline{Transpose}{\trans}{\m{A}\trans}
    \docline{Trace}{\tr}{\tr(\m{A})}
\end{doctable}

\section{Calculus}
\begin{doctable}
    \docline{Full derivative}{\fd}{\fd{y}{x}}
    \docline{Full second derivative}{\fdd}{\fdd{y}{x}}
    \docline{Full $n$th derivative}{\fd}{\fd[5]{y}{x}}
    \docline{Partial derivative}{\pd}{\pd{y}{x}}
    \docline{Partial second derivative}{\pdd}{\pdd{y}{x}}
    \docline{Partial $n$th derivative}{\pd}{\pd[5]{y}{x}}
    \docline{Material derivative}{\Dt}{\Dt{x}}
\end{doctable}

\end{document}
